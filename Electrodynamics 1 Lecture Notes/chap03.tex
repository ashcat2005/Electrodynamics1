\chapter{Boundary Value Problems in Electrostatics} 

\section{ Method of Images}

The \textit{Method of Images} works for problems in which there is one (or more) point charges in the presence of boundary surfaces. The method is based in the fact that, under some favorable conditions, it is possible to infer from the given geometry that a small number of charges, called \textit{image charges}, with appropriate magnitudes and location (out of the region of interest) can simulate the required boundary conditions. Since the image charges are external to the volume of interest, their potentials are solution of Laplace's equation inside that volume.

\subsection{ Point Charge in the Presence of a Grounded Infinite Conductor Plane}
As shown in the Figure, consider a point charge located at a position $\textbf{r}' = \boldsymbol{\ell} = \ell \textbf{e}_1$ in front of a vertical infinite conductor plane at zero potential. Hence, the Dirichlet boundary condition are $\Phi=0$ at the points in the line $x=0$ and $\Phi=0$ at $\{ x\rightarrow \infty, y\rightarrow \pm \infty ,z \rightarrow \pm \infty \} $. From the geometry, it is easy to infer that this problem is equivalent to the problem of the original point charge together with an equal and opposite  charge  located at the mirror-image point $\textbf{r}'_I = -\boldsymbol{\ell}$, behind the conductor surface. Therefore, the potential at any point $\textbf{r}$ is given by the superposition
\begin{equation}
\Phi (\textbf{r}) = \frac{1}{4\pi \epsilon_0} \frac{q}{\left| \textbf{r} - \textbf{r}' \right|} - \frac{1}{4\pi \epsilon_0} \frac{q}{\left| \textbf{r} - \textbf{r}'_I \right|}.
\end{equation}

Here we identify the Green function
\begin{equation}
G(\textbf{r}, \textbf{r}') = \frac{1}{\left| \textbf{r} - \textbf{r}' \right|} -  \frac{1}{\left| \textbf{r} - \textbf{r}'_I \right|}
\end{equation}
in the region with $x\geq 0$. Note that this functions satisfies the imposed Dirichlet boundary conditions . It is also possible to identify that the method of images fixes the function $F(\textbf{r}, \textbf{r}')$ as
\begin{equation}
F(\textbf{r}, \textbf{r}') = -  \frac{1}{\left| \textbf{r} - \textbf{r}'_I \right|}.
\end{equation}


\subsection{ Finite Line of Charge}


\subsection{ Point Charge in the Presence of a Grounded Conducting Sphere}

Consider a point charge $q$ located at the position $\textbf{r}$ relative to the origin, around which is centered a grounded conducting sphere of radius $R$. We want to find the potential $\Phi(\textbf{r})$ subject to the boundary conditions $\Phi (r=R) =0$ and $\Phi (r\rightarrow \infty) =0$. We will assume that it is needed only one image charge $q_I$ lying on the ray going from the origin to the charge $q$. If $q$ lies outside the sphere, the position $\textbf{r}'_I$ of the image charge lies inside the sphere. Hence the potential due to both charges is given by
\begin{equation}
\Phi (\textbf{r}) = \frac{1}{4\pi \epsilon_0} \frac{q}{\left| \textbf{r} - \textbf{r}' \right|} + \frac{1}{4\pi \epsilon_0} \frac{q_I}{\left| \textbf{r} - \textbf{r}'_I \right|}, \label{ImageExample1}
\end{equation}
but we must chose $q_I$ and $\textbf{r}'_I$ to satisfy the boundary conditions. Define the unit vectors $\textbf{n}$ along the direction $\textbf{r}$ and $\textbf{n}'$ along $\textbf{r}'$. The we have  
\begin{equation}
\Phi (\textbf{r}) = \frac{1}{4\pi \epsilon_0} \frac{q}{\left| r \textbf{n} - r' \textbf{n}' \right|} + \frac{1}{4\pi \epsilon_0} \frac{q_I}{\left| r \textbf{n} - r'_I \textbf{n}' \right|}.  
\end{equation}

Factorizing $r$ in the first term and $r'_I$ in the second, we obtain
\begin{equation}
\Phi (\textbf{r}) = \frac{1}{4\pi \epsilon_0} \frac{q}{r \left|  \textbf{n} -\frac{ r'}{r} \textbf{n}' \right|} + \frac{1}{4\pi \epsilon_0} \frac{q_I}{r'_I\left| \frac{r}{r'_I} \textbf{n} - \textbf{n}' \right|},
\end{equation}
and evaluating at the conductor sphere, $r=R$, yields
\begin{equation}
\Phi (r=R) = \frac{1}{4\pi \epsilon_0} \left[ \frac{q}{R \left|  \textbf{n} -\frac{ r'}{R} \textbf{n}' \right|} + \frac{q_I}{r'_I\left| \frac{R}{r'_I} \textbf{n} - \textbf{n}' \right|} \right] = 0.
\end{equation}

In order to satisfy this boundary condition, we choose
\begin{equation}
\frac{q_I}{r'_I} = -\frac{q}{R} 
\end{equation}
and
\begin{equation}
\frac{R}{r'_I} = \frac{r'}{R}.
\end{equation}

These relations are combined to obtain the image charge
\begin{equation}
 q_I = -\frac{R}{r'} q
 \end{equation}
and its position
\begin{equation}
r'_I = \frac{R^2}{r'}.
\end{equation}

If the charge $q$ is brought closer to the sphere (i.e. $r' \rightarrow R$) the image charge grows in magnitude, $q_I \rightarrow -q$, and it is moves to the center of the sphere, $r'_I \rightarrow R$. \\
The surface charge density in the conducting sphere is calculated using the image charge and the normal derivative of the potential at the surface (remember that $ \textbf{E}  \cdot \textbf{n}= -\boldsymbol{\nabla} \Phi \cdot \textbf{n} = \frac{\sigma}{\epsilon_0}$),
\begin{equation}
 \sigma = -\epsilon_0 \left. \boldsymbol{\nabla} \Phi \cdot \textbf{n}  \right|_{r=R} = - \frac{1}{4 \pi } \left. \boldsymbol{\nabla} \left[  \frac{q}{\left| r \textbf{n} - r' \textbf{n}' \right|} + \frac{q_I}{\left| r \textbf{n} - r'_I \textbf{n}' \right|} \right] \cdot \textbf{n}  \right|_{r=R}   
 \end{equation} 
\begin{equation}
 \sigma =  \frac{1}{4 \pi } \left. \left[  \frac{q (r \textbf{n} - r' \textbf{n}' ) \cdot \textbf{n} }{\left| r \textbf{n} - r' \textbf{n}' \right|^3 } + \frac{q_I (r \textbf{n} - r'_I \textbf{n}')\cdot \textbf{n} }{\left| r \textbf{n} - r'_I \textbf{n}' \right|^3} \right]  \right|_{r=R}   
 \end{equation} 
\begin{equation}
 \sigma =  \frac{1}{4 \pi } \left. \left[  \frac{q r }{\left| r \textbf{n} - r' \textbf{n}' \right|^3 } + \frac{q_I r }{\left| r \textbf{n} - r'_I \textbf{n}' \right|^3} \right]  \right|_{r=R}   
 \end{equation} 
\begin{equation}
 \sigma =  \frac{1}{4 \pi } \left. \left[  \frac{q r }{\left(r^2 + r'^2 - 2 rr' \textbf{n} \cdot \textbf{n}' \right)^{3/2} } + \frac{q_I r }{\left(r^2 + r'^2_I - 2 rr'_I \textbf{n} \cdot \textbf{n}' \right)^{3/2}} \right]  \right|_{r=R}   
 \end{equation} 
\begin{equation}
 \sigma =  \frac{1}{4 \pi } \left[  \frac{q R }{\left(R^2 + r'^2 - 2 Rr' \textbf{n} \cdot \textbf{n}' \right)^{3/2} } + \frac{q_I R }{\left(R^2 + r'^2_I - 2 Rr'_I \textbf{n} \cdot \textbf{n}' \right)^{3/2}} \right].  
 \end{equation} 
Replacing the values of $q_I$ and $r'_I$ we have
\begin{equation}
 \sigma =  \frac{qR}{4 \pi } \left[  \frac{ 1 }{\left(R^2 + r'^2 - 2 Rr' \textbf{n} \cdot \textbf{n}' \right)^{3/2} } - \frac{\frac{R}{r'}  }{\left(R^2 + \left( \frac{R^2}{r'} \right)^2 - 2 R \left( \frac{R^2}{r'} \right)  \textbf{n} \cdot \textbf{n}' \right)^{3/2}} \right]  
 \end{equation} 
\begin{equation}
 \sigma =  \frac{qR}{4 \pi } \left[  \frac{ 1 }{r'^3 \left(\left(\frac{R}{r'}\right)^2 + 1 - 2 \frac{R}{r'} \textbf{n} \cdot \textbf{n}' \right)^{3/2} } - \frac{\frac{R}{r'}  }{R^3 \left(1 + \left( \frac{R}{r'} \right)^2 - 2  \left( \frac{R}{r'} \right)  \textbf{n} \cdot \textbf{n}' \right)^{3/2}} \right]
 \end{equation} 
\begin{equation}
 \sigma =  \frac{qR}{4 \pi } \frac{ 1 }{ \left( 1+ \left(\frac{R}{r'}\right)^2 - 2 \frac{R}{r'} \textbf{n} \cdot \textbf{n}' \right)^{3/2} } \left[  \frac{ 1 }{r'^3 } - \frac{1 }{R^2 r'} \right]  
 \end{equation} 
 \begin{equation}
 \sigma =  \frac{qR}{4 \pi } \frac{R}{r'} \frac{ 1 }{ \left( 1+ \left(\frac{R}{r'}\right)^2 - 2 \frac{R}{r'} \textbf{n} \cdot \textbf{n}' \right)^{3/2} } \left[  \frac{ 1 }{r'^2 R } - \frac{1 }{R^3} \right] 
 \end{equation} 
  \begin{equation}
 \sigma = - \frac{q}{4 \pi R^2 } \frac{R}{r'} \frac{ \left(  1- \frac{ R^2}{r'^2  }  \right) }{ \left( 1+ \left(\frac{R}{r'}\right)^2 - 2 \frac{R}{r'} \textbf{n} \cdot \textbf{n}' \right)^{3/2} } .  
 \end{equation} 
 
Defining $\textbf{n} \cdot \textbf{n}' = \cos \gamma$ we write
  \begin{equation}
 \sigma = - \frac{q}{4 \pi R^2 } \frac{R}{r'} \frac{ \left(  1- \frac{ R^2}{r'^2  }  \right) }{ \left( 1+ \left(\frac{R}{r'}\right)^2 - 2 \frac{R}{r'} \cos \gamma \right)^{3/2} } .  
 \end{equation} 

The force acting on $q$ is calculated using the image charge $q_I$ and the distance between them: 
\begin{equation}
r_{qq_I} = r' - r'_I = r' \left( 1 - \frac{R^2}{r'^2}\right).
\end{equation}

Hence, this force is simply
\begin{equation}
\textbf{F} = \frac{1}{4\pi \epsilon_0} \frac{qq_I}{r_{qq_I}^2} \textbf{n}
\end{equation}
\begin{equation}
\textbf{F} = \frac{1}{4\pi \epsilon_0} \frac{q \left( -\frac{R}{r'} q \right)}{r'^2 \left( 1 - \frac{R^2}{r'^2}\right)^2} \textbf{n}
\end{equation}
\begin{equation}
\textbf{F} = - \frac{1}{4\pi \epsilon_0} \frac{q^2}{R^2}  \left( \frac{R}{r'}  \right)^3 \left( 1 - \frac{R^2}{r'^2}\right)^{-2} \textbf{n}.\label{ForceImage1}
\end{equation}

If the point charge $q$ lies inside the sphere, the the same results apply.

\subsection{ Point Charge in the Presence of a Charged, Insulated, Conducting Sphere}

In order to obtain the potential due to a point charge $q$ near a charged, insulated, conducting sphere we begin with the result of the previous section in which the grounded conducting sphere acquires a total charge of $q_I$ distributed on its surface. The the ground is disconnected and we add to the sphere the charge $(Q-q_I)$ which will bring the total charge of the sphere up to $Q$. However, we may think as if this added charge will simply distribute uniformly over the surface of the sphere (because the external point charge $q$ is already balanced by the image $q_I$). Therefore,  the total potential is obtained by adding the potential found in the previous section, (\ref{ImageExample1}), and the potential of a point charge $(Q-q_I)$ at the origin. This gives
\begin{equation}
\Phi (\textbf{r}) = \frac{1}{4\pi \epsilon_0} \frac{q}{\left| \textbf{r} - \textbf{r}' \right|} + \frac{1}{4\pi \epsilon_0} \frac{q_I}{\left| \textbf{r} - \textbf{r}'_I \right|} +  \frac{1}{4\pi \epsilon_0} \frac{Q-q_I}{\left| \textbf{r} \right|}
\end{equation}
\begin{equation}
\Phi (\textbf{r}) = \frac{1}{4\pi \epsilon_0} \left[ \frac{q}{\left| \textbf{r} - \textbf{r}' \right|} +  \frac{q_I}{\left| \textbf{r} - \textbf{r}'_I \right|} +   \frac{Q-q_I}{\left| \textbf{r} \right|} \right]
\end{equation}
\begin{equation}
\Phi (\textbf{r}) = \frac{1}{4\pi \epsilon_0} \left[ \frac{q}{\left| \textbf{r} - \textbf{r}' \right|} -  \frac{R q}{r' \left| \textbf{r} - \frac{R^2}{r'^2}\textbf{r}' \right|} +   \frac{Q+\frac{R}{r'}q}{\left| \textbf{r} \right|} \right].
\end{equation}

Similarly, the force acting on the charge $q$ is obtained by superposition of equation (\ref{ForceImage1}) with the Coulomb  force produced by $(Q-q_I)$ ,
\begin{equation}
\textbf{F} = -\frac{1}{4\pi \epsilon_0} \frac{q^2}{R^2}  \left( \frac{R}{r'}  \right)^3 \left( 1 - \frac{R^2}{r'^2}\right)^{-2} \textbf{n} + \frac{1}{4\pi \epsilon_0} \frac{q(Q-q_I)}{r'^2} \textbf{n}
\end{equation}
\begin{equation}
\textbf{F} = - \frac{1}{4\pi \epsilon_0} \frac{q^2}{R^2}  \left( \frac{R}{r'}  \right)^3 \left( 1 - \frac{R^2}{r'^2}\right)^{-2} \textbf{n} + \frac{1}{4\pi \epsilon_0} \frac{q(Q+ \frac{R}{r'}q)}{r'^2} \textbf{n}
\end{equation}
\begin{equation}
\textbf{F} = \frac{1}{4\pi \epsilon_0}  \frac{q}{r'^2} \left[Q - q \left( \frac{R}{r'}  \right) \left( 1 - \frac{R^2}{r'^2}\right)^{-2} + \frac{R}{r'}q \right] \textbf{n}
\end{equation}
\begin{equation}
\textbf{F} = \frac{1}{4\pi \epsilon_0}  \frac{q}{r'^2} \left[Q - q \left( \frac{R}{r'}  \right) \left(   \left( 1 - \frac{R^2}{r'^2}\right)^{-2} -1 \right) \right] \textbf{n}
\end{equation}
\begin{equation}
\textbf{F} = \frac{1}{4\pi \epsilon_0}  \frac{q}{r'^2} \left[ Q - q \left( \frac{R}{r'} \right)  \frac{1 - \left(1 - \frac{R^2}{r'^2}\right)^{2} }{ \left( 1 - \frac{R^2}{r'^2}\right)^2 }   \right] \textbf{n}
\end{equation}
\begin{equation}
\textbf{F}= \frac{1}{4\pi \epsilon_0}  \frac{q}{r'^2} \left[ Q - q \left( \frac{R}{r'} \right)  \frac{r'^4 - \left(r'^2 - R^2 \right)^{2} }{ \left( r'^2 - R^2 \right)^2 }   \right] \textbf{n}
\end{equation}
\begin{equation}
\textbf{F}= \frac{1}{4\pi \epsilon_0}  \frac{q}{r'^2} \left[ Q - q \left( \frac{R}{r'} \right)  \frac{ 2 r'^2 R^2 - R^4  }{ \left( r'^2 - R^2 \right)^2 }   \right] \textbf{n}
\end{equation}
\begin{equation}
\textbf{F}= \frac{1}{4\pi \epsilon_0}  \frac{q}{r'^2} \left[ Q - q \left( \frac{R^3}{r'} \right)  \frac{ 2 r'^2 - R^2  }{ \left( r'^2 - R^2 \right)^2 }   \right] \textbf{n}.
\end{equation}

Note that in the linimt $r' \gg R$ the force reduces to the Coulomb's law,
\begin{equation}
\lim_{r' \gg R} \textbf{F}  =  \frac{1}{4\pi \epsilon_0}  \frac{qQ}{r'^2} \textbf{n}
\end{equation}





Hence, the laplacian in spherical coordinates is written
\begin{equation}
\nabla^2 \Phi = \frac{1}{r^2}\frac{\partial}{\partial r} \left( r^2 \frac{\partial \Phi}{\partial r} \right) + \frac{1}{r^2 \sin \theta} \frac{\partial}{\partial \theta} \left( \sin \theta \frac{\partial \Phi}{\partial \theta} \right)+ \frac{1}{r^2 \sin^2 \theta} \frac{\partial ^2 \Phi}{\partial \varphi ^2} 
\end{equation}